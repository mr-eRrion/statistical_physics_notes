% !TeX root = 统计物理.tex

\section{气体动理学}
气体动力学, aka气体的非平衡态演化理论,是从经典的角度描述大量分子组成的气体的性质.
\subsection{细说刘维尔定理}
    省流: 相空间内的相点的演化和不可压缩气流相同.(这个不难验证, 对相体积作用时间演化即可.)

    连续性方程给出;
    \[\dfrac{\partial \rho}{\partial t}+\nabla \cdot (\rho v)=0\]
    利用哈密顿正则方程:
    \[\dot{q}_{i}=\dfrac{\partial H}{\partial p_{i}} \qquad \dot{p}_{i}=-\dfrac{\partial H}{\partial q_{i}}\]
    于是可以得到($s$代表自由度数):
    \[\nabla \cdot (\rho v)=\sum_{i=1}^{s}\left[ \dot{q}_{i}\dfrac{\partial \rho}{\partial q_{i}}+\dot{p}_{i}\dfrac{\partial \rho}{\partial p_{i}} \right] +\rho\sum_{i=1}^{s}\left[ \dfrac{\partial \dot{q}_{i}}{\partial q_{i}}+\dfrac{\partial \dot{p}_{i}}{\partial p_{i}} \right] =\sum_{i=1}^{s}\left[ \dot{q}_{i}\dfrac{\partial \rho}{\partial q_{i}}+\dot{p}_{i}\dfrac{\partial \rho}{\partial p_{i}} \right] \]
    于是 
    \begin{equation}
      \dfrac{\partial\rho}{\partial t}=\dfrac{\mathrm{d}  \rho}{\mathrm{d}  t}-\sum_{i=1}^{s}\left( \dfrac{\partial \rho}{\partial q_{i}}\dot{q}_{i}+\dfrac{\partial \rho}{\partial p_{i}}\dot{p}_{i} \right) = -\sum_{i=1}^{s}\left( \dfrac{\partial \rho}{\partial q_{i}}\dot{q}_{i}+\dfrac{\partial \rho}{\partial p_{i}}\dot{p}_{i} \right)
    \end{equation}

    我们继续引入泊松括号:$[\alpha,\beta]=\sum \left( \dfrac{\partial \alpha}{\partial q_{i}}\cdot \dfrac{\partial \beta}{\partial p_{i}}-\dfrac{\partial \alpha}{\partial p_{i}}\cdot \dfrac{\partial \beta}{\partial q_{i}} \right)$. 
    
    \vspace*{0.2cm}
    
    把刘维尔定理写成:
    \begin{equation}
      \dfrac{\partial \rho}{\partial t}=-[\rho,H]
    \end{equation}

    刘维尔定理的一些引理:
    \begin{itemize}
      \item[1.]物理量的系综平均在时间的演化写成:
        \begin{equation}
          \dfrac{\mathrm{d}\left< A \right> }{\mathrm{d} t}=\int \mathrm{d} \Gamma \dfrac{\partial \rho}{\partial t}A=\sum_{i=1}^{s}\mathrm{d} \Gamma A(p,q)\left( \dfrac{\partial \rho}{\partial p_{i}}\dfrac{\partial H}{\partial q_{i}}-\dfrac{\partial \rho}{\partial q_{i}}\dfrac{\partial H}{\partial p_{i}} \right) 
        \end{equation}
        对上面的积分式进行分部积分,而$\rho$在边界上为零测.即$\displaystyle \int f\rho'=-\int f' \rho$,代入上式得到 
        \begin{align}
          \dfrac{\mathrm{d}\left< A \right> }{\mathrm{d} t} &= -\sum_{i=1}^{s} \int \mathrm{d} \Gamma \rho\left[ \left( \dfrac{\partial A}{\partial p_{i}}\cdot \dfrac{\partial H}{\partial q_{i}}-\dfrac{\partial A}{\partial q_{i}}\dfrac{\partial H}{\partial p_{i}} \right) +A \left( \frac{\partial^2 H}{\partial p_{i} \partial q_{i}}-\frac{\partial^2 H}{\partial p_{i} \partial q_{i}} \right)  \right]\\
          &=-\int \mathrm{d} \Gamma \rho [H,A]=\int \mathrm{d} \Gamma \rho [A,H] 
        \end{align}
        并注意不能轻易地把$\dfrac{\mathrm{d}}{\mathrm{d} t}$放到积分里面去.
      \item[2.]
        考察平衡态的刘维尔定理,有
        \[[\rho,H]=0\]
        注意到这个方程有解:$\rho=\rho(H)$, 它表明$\rho$在等能面上处处相等. 这就是统计力学基本假设: 等概率原理. 当然同时,$\rho$也可以含有系统的运动积分,它并不破坏系统的一切守恒原理.

    \end{itemize}
\subsection{BBGKY方程链}
    所谓BBGKY,其实是五个人的名字首字母(Bogoloubov-Born-Green-Kirkwood-Yvon),而BBGKY方程链,则是把$s$个粒子的分布函数的时间演化用$s+1$个粒子的分布函数表示出来的方程.\footnote{注:下面不出意外所有$q_{i}$均代表$\vec{q_{i}}$.相应的$\dfrac{\partial }{\partial q}$为一个矢量,而$\mathrm{d} q$代表多重积分}

    只考虑二体相互作用以及外场的简单情形下,哈密顿量写成
    \[H=\sum_{i=1}^{N}\left( \frac{p_{i}^{2}}{2m}+U(q_{i}) \right) +\frac{1}{2}\sum_{i\neq j} V(q_{i}-q_{j})\]
    接下来定义$f_{s}$,它是有任意$s$个粒子处在$(p_1,q_1),(p_2,q_2)\cdots$的概率分布
    \[f_{s}(p_1,q_1 \cdots p_{s}, q_{s},t)=\frac{N!}{(N-s)!} \int \prod_{i=s+1}^{N} \mathrm{d} \Gamma_{i} \rho(p,q,t)=\frac{N!}{(N-s)!}\rho_{s}\]

    考虑$f_{s}$的时间演化,把哈密顿量分成三部分:
    \[H=H_{s}+H_{N-s}+H'\]
    其中
    \[H_{s}=\sum_{i=1}^{s}\left( \frac{p_{i}^{2}}{2m}+U(q_{i}) \right) +\frac{1}{2}\sum_{i\neq j\le s}V(q_{i}-q_{j})\]
    \[H_{N-s}=\sum_{i=s+1}^{N}\left( \frac{p_{i}^{2}}{2m}+U(q_{i}) \right) +\frac{1}{2}\sum_{i\neq j> s}V(q_{i}-q_{j})\]
    还有
    \[H'=\sum_{i=1}^{s}\sum_{j=s+1}^{N}V(q_{i}-q_{j})\]

    $\rho_{s}$的时间演化为:
    \begin{equation}
      \dfrac{\partial \rho_{s}}{\partial t}=\int \prod_{i=s+1}^{N} \mathrm{d} \Gamma_{i} \dfrac{\partial \rho}{\partial t}=-\int \prod_{i=s+1}^{N} [\rho,H_{s}+H_{N-s}+H']
    \end{equation}
    分别计算几个泊松括号:
    
    第一项因为$H_{s}$和后$s+1$个坐标无关,于是积分可以放到泊松括号内:
    \[\int \prod_{i=s+1}^{N} \mathrm{d}\Gamma_{i}[\rho,H_{s}]=\left[ \int \prod_{i=s+1}^{N} \rho~\mathrm{d}\Gamma_{i}  ~,~ H_{s} \right]=[f_{s},H_{s}] \]
    
    第二项,写出其完整表达式
    \begin{align*}
      -\int \prod_{i=s+1}^{N} \mathrm{d} \Gamma_{i} [\rho,H_{N-s}]&=\int \prod_{i=s+1}^{N} \mathrm{d} \Gamma_{i} \sum_{j=1}^{N}\left[ \dfrac{\partial \rho}{\partial p_{j}}\dfrac{\partial H_{N-s}}{\partial q_{j}}-\dfrac{\partial \rho}{\partial q_{j}}\dfrac{\partial H_{N-s}}{\partial p_{j}} \right] 
      \\ 
      &=\int \prod_{i=s+1}^{N} \mathrm{d} \Gamma_{i} \sum_{j=s+1}^{N} \left[ \dfrac{\partial \rho}{\partial p_{j}}\left( \dfrac{\partial U}{\partial q_{j}}+\frac{1}{2}\sum_{k=s+1}^{N} \dfrac{\partial V(q_{j}-q_{k})}{\partial q_{j}} \right) -\dfrac{\partial \rho}{\partial q_{j}}\frac{p_{j}}{m}
      \right]
      \\
      &=0 
    \end{align*}
    最后一个等号利用了分部积分,$\dfrac{\partial \rho}{\partial p_{j}}$的因子和$p_{j}$无关,而$\dfrac{\partial p}{\partial q_{j}}$因子和$q_{j}$无关.而且边界上$\rho$积分消失.

    第三项:
    \begin{align*}
      &\int \prod_{i=s+1}^{N} \mathrm{d} \Gamma_{i} \sum_{j=1}^{N}\left[ \dfrac{\partial \rho}{\partial p_{j}}\dfrac{\partial H'}{\partial q_{j}}-\dfrac{\partial \rho}{\partial p_{j}}\dfrac{\partial H'}{\partial p_{j}} \right]\\
      =&\int \prod_{i=s+1}^{N} \mathrm{d} \Gamma_{i} \left[ \sum_{n=1}^{s}\dfrac{\partial \rho}{\partial p_{n}}\sum_{j=s+1}^{N} \dfrac{\partial V(q_{n}-q_{j})}{\partial q_{n}}+\sum_{j=s+1}^{N} \dfrac{\partial \rho}{\partial p_{j}}\sum_{n=1}^{s}\dfrac{\partial V(q_{j}-q_{n})}{\partial q_{j}} \right]  
    \end{align*}
    这里利用第一个等号到了$H'$和$p_{j}$无关的性质.上式分部积分得到后一项为零.而前一项对于$j>s+1$的指标是对称的,于是原式成为
    \[(N-s)\int \prod_{i=s+1}^{N} \mathrm{d} \Gamma_{i} \sum_{j=1}^{s}\dfrac{\partial V(q_{j}-q_{s+1})}{\partial q_{j}}\cdot \dfrac{\partial \rho}{\partial p_{j}}\]
    \[=(N-s)\sum_{j=1}^{s}\int \mathrm{d} \Gamma_{s+1} \dfrac{\partial V(q_{j}-q_{s+1})}{\partial q_{j}}\cdot \dfrac{\partial }{\partial p_{j}}\left[ \int \prod_{i=s+2}^{N} \mathrm{d} \Gamma_{i}\rho \right] \]
    注意到后一个方程方括号内的内容就是$\rho_{s+1}$,那么
    \begin{equation}
      \dfrac{\partial \rho_{s}}{\partial t}-[H_{s},\rho_{s}]=(N-s)\sum _{j=1}^{s} \int \mathrm{d}  \Gamma_{s+1} \dfrac{\partial V(q_{j}-q_{s+1})}{\partial q_{n}}\cdot \dfrac{\partial \rho_{s+1}}{\partial p_{n}}
    \end{equation}
    把$\rho_{s}$换成$f_{s}$,得到:
    \begin{equation}
      \dfrac{\partial f_{s}}{\partial t}-[H_{s},f_{s}]=\sum _{j=1}^{s} \int \mathrm{d}  \Gamma_{s+1} \dfrac{\partial V(q_{j}-q_{s+1})}{\partial q_{n}}\cdot \dfrac{\partial f_{s+1}}{\partial p_{n}}
      \label{eq:6.8}
    \end{equation}
    这就是BBGKY方程链.
    
    注意到如果相互作用不存在,则$f_{s}$的演化和不可压缩流体一样, 这也正是刘维尔定理成立需要的要求,即体系是闭合的.

    这个方程讲述了这样一件事情:在一个$s$个粒子的系统中,它的分布函数$f_{s}$的时间演化,由自身系统的演化 和 另外$N-s$个粒子对$s$粒子的碰撞项构成. 而这个碰撞项的碰撞积分(\eqref{eq:6.8}的右端),由对子系统的$s$个粒子求和任意一个另外的粒子产生的碰撞项.

    但实际上这个方程链本身特别复杂,我们需要一些物理上的近似来简化.

\subsection{玻尔兹曼方程}
    现在需要考虑BBGKY方程链的近似问题. 为了了解各个项之间的大小关系, 利用室温下的平均速度和尺寸对一些时间量做估计.
    \begin{itemize}
      \item[1.]第一个估计是外场估计,源于
        \[\frac{1}{\tau_{U}} \sim \dfrac{\partial U}{\partial q}\dfrac{\partial }{\partial p}\]
        是外场发挥明显作用的时间尺度
      \item[2.]第二个是内部相互作用的估计,源于
        \[\frac{1}{\tau_{c}} \sim \dfrac{\partial V}{\partial q}\dfrac{\partial }{\partial p} \]
        是两个粒子处在相互作用范围内的特征时间.这一般是问题中会涉及的最小的特征时间. 但在长程相互作用是会变得更复杂,比如等离子体中的库伦相互作用.
      \item[3.]第三个是碰撞时间估计,起源于碰撞积分,
        \[\frac{1}{\tau_{\times}} \sim  \int \mathrm{d} \Gamma ~\dfrac{\partial V}{\partial q}\cdot\dfrac{\partial }{\partial p} \frac{f_{s+1}}{f_{s}} \sim  \int \mathrm{d} \Gamma ~\dfrac{\partial V}{\partial q}\cdot \dfrac{\partial }{\partial p} ~N \frac{\rho_{s+1}}{\rho_{s}}\]
        这个积分只会在相互作用区域内明显的非零. 而$\dfrac{f_{s+1}}{f_{s}}$和在单位体积内找到另一个粒子的概率. 记粒子数密度为$n$,相互作用的特征尺度为$d$, 则$\tau_{\times}$的估计为:
        \[\tau_{\times} \approx  \frac{\tau_{c}}{nd^{3}} \approx  \frac{1}{nvd^{2}}\]
        被称为平均自由时(对应于平均自由程)
    \end{itemize}

    玻尔兹曼方程由BBGKY中假设短程相互作用和稀薄气体($\tau_{c} / \tau_{\times} \approx  nd^{3} \ll 1$)
        
    写出BBGKY方程链中(对玻尔兹曼方程而言)最重要的两个方程:
    \begin{equation}
      \left[ \dfrac{\partial }{\partial t}-\dfrac{\partial U}{\partial q_1}\cdot \dfrac{\partial }{\partial p_1} + \frac{p_1}{m}\cdot \dfrac{\partial }{\partial q_1} \right] f_1=\int \mathrm{d} \Gamma_2 \dfrac{\partial V(q_1-q_2)}{\partial q_1} \dfrac{\partial f_2}{\partial p_1}
      \label{eq:6.9}
    \end{equation}
    \[
      \left[ \dfrac{\partial }{\partial t}-\dfrac{\partial U}{\partial q_1}\cdot \dfrac{\partial }{\partial p_1}-\dfrac{\partial U}{\partial q_2}\cdot \dfrac{\partial }{\partial p_2}+\frac{p_1}{m} \cdot \dfrac{\partial }{\partial q_1}+\frac{p_2}{m} \cdot \dfrac{\partial }{\partial q_2}-\dfrac{\partial V(q_1-q_2)}{\partial q_1}(\dfrac{\partial }{\partial p_1}-\dfrac{\partial }{\partial p_2}) \right] f_2
    \]
    \begin{equation}
      =\int \mathrm{d} \Gamma_3 \left[ \dfrac{\partial V(q_1-q_3)}{\partial q_1} \cdot \dfrac{\partial }{\partial p_1}+ \dfrac{\partial V(q_2-q_3)}{\partial q_2} \cdot \dfrac{\partial }{\partial p_2} \right] f_3
    \end{equation}
    在稀薄气体情形下,可以合理的认为同时最多只有两个粒子在发生相互作用, 此时其他粒子是独立的,即 
    \[\rho_2(p_1,q_1,p_2,q_2,t) \to  \rho_1(p_1,q_1,t)\rho_1(p_2,q_2,t)\]
    或
    \[f_2(p_1,q_1,p_2,q_2,t) \to  f_1(p_1,q_1,t)f_1(p_2,q_2,t) \quad \text{when}\left\vert q_2-q_1 \right\vert \gg d\]

    为了描述一个二粒子的碰撞过程(终极目的是计算\eqref{eq:6.9}的右侧):

    考察比$\tau_{c}$略长的时间尺度(指比$\tau_{U}$短), 此时短程力下的$f_2$具有"稳态"性质. 于是忽略掉小项, 在此情境下, BBGKY第二个方程简化为:
    \begin{equation}
      \left[ \frac{p_1}{m} \cdot \dfrac{\partial }{\partial q_1} + \frac{p_2}{m} \cdot \dfrac{\partial }{\partial q_2}- \dfrac{\partial V(q_1-q_2)}{\partial q_1}\cdot \left( \dfrac{\partial }{\partial p_1}-\dfrac{\partial }{\partial p_2} \right)  \right] f_2=0
    \end{equation}
    我们期望$f_2(q_1,q_2)$能表现出质心的慢变和相对坐标的快变(这才是碰撞!),引入相对坐标$q=q_2-q_1$,于是 
    \begin{equation}
      \dfrac{\partial V(q_1-q_2)}{\partial q_1}\cdot \left( \dfrac{\partial }{\partial p_1}-\dfrac{\partial }{\partial p_2} \right) f_2=-\left(\frac{p_1-p_2}{m}\right)\dfrac{\partial }{\partial q}f_2
    \end{equation}
    这样,\eqref{eq:6.9}的右侧可以写成
    \begin{equation}
      \int \mathrm{d} p_2 \mathrm{d} q_2 \dfrac{\partial V(q_1-q_2)}{\partial q_1} \cdot \left( \dfrac{\partial }{\partial p_1}-\dfrac{\partial }{\partial p_2} \right) f_2 \approx \int \mathrm{d} p_2 \mathrm{d} q \left( \frac{p_2-p_1}{m} \right) \cdot \dfrac{\partial }{\partial q}f_2(p_1,q_1,p_2,q;t)
    \end{equation}
    这里,从\eqref{eq:6.9}右侧到上式左侧,利用了
    \[\int \mathrm{d} \Gamma_2 \dfrac{\partial V(q_1-q_2)}{\partial q_1} \dfrac{\partial f_2}{\partial p_2}=0\]
    这件事(同样,分部积分即可). 而后一个约等号来自于质心慢变而相对坐标快变情况, 即$\dfrac{\partial f_2}{\partial q_2} \approx -\dfrac{\partial f_2}{\partial q_1} \approx \dfrac{\partial f_2}{\partial q}$.
    
    \vspace*{0.5cm}
        
    接下来就要对碰撞的过程做一些分析了:
    
    先对一些物理量做一个澄清:我们选取质心系作为参考系, 在考察碰撞的这一段时间内, 质心将不会带来显著的影响. 采取平行与相对动量的坐标系, 把相对坐标内平行于相对动量的坐标记为$a$,另一部分记为$b$ (应该是$\vec{b}$,这里和前面一样省略所有矢量记号). 坐标轴方向的选取则使得碰撞前$a$是负数而碰撞后是正数.
    \begin{center}
      \includegraphics*[width=0.55\linewidth]{./picture/Boltzmann.png}
    \end{center}
    
    先把对$a$的积分积掉,利用$\dfrac{p_1-p_2}{m}=v_1-v_2$,而$a$在碰撞前负,碰撞后正.则碰撞积分可以写成:
    \[\int \mathrm{d} p_2 \mathrm{d} b \left\vert v_1-v_2 \right\vert \left[ f_2(p_1,q_1,p_2,b,+;t)-f_2(p_1,q_1,p_2,b,-;t) \right] \]

    由时间反演对称性,$f_2(p_1',q_1,p_2',b,-;t)=f_2(p_1,q_1,p_2,b,+;t)$,这里$p_1',p_2'$是出射的动量. 在考虑的碰撞是弹性的情况下, 动量仅仅改变一个偏角而不改变大小. 注意到$\mathrm{d} b$其实是散射截面,而我们把积分因子换到立体角$\mathrm{d} \Omega$上去, 于是我们做一个变换:
    \begin{equation}
      \dfrac{\mathrm{d}f_1}{\mathrm{d} t}=\int \mathrm{d} p_2 \mathrm{d} \Omega \left\vert \dfrac{\mathrm{d}\sigma}{\mathrm{d} \Omega} \right\vert \left\vert v_1-v_2 \right\vert \left[ f_2(p_1',q_1,p_2',b,-;t)-f_2(p_1,q_1,p_2,b,-;t) \right] 
    \end{equation}
    这里$\dfrac{\mathrm{d}\sigma}{\mathrm{d} \Omega}$就是我们熟悉的微分散射截面了.比如对于库伦势,它为:
    \[\left\vert \dfrac{\mathrm{d}\sigma}{\mathrm{d} \Omega} \right\vert =\left( \frac{me^{2}}{2\left\vert p \right\vert ^{2}\sin ^{2}(\theta / 2)} \right) ^{2}\]
    接下来引入假设:
    \begin{equation}
      f_2(p_1,q_1,p_2,b,-;t)=f_1(p_1,q_1;t)\cdot f_1(p_2,q_2;t)
    \end{equation}
    这被称为\emph{分子混沌假设},即在很远的区域分子是没有关联的. 那么我们的最终方程被写为:
    \begin{equation}
      \left[ \dfrac{\partial }{\partial t}-\dfrac{\partial U}{\partial q_1}\cdot \dfrac{\partial }{\partial p_1}+\frac{p_1}{m}\cdot \dfrac{\partial }{\partial q_1} \right] f_1=-\int \mathrm{d} p_2 \mathrm{d} \Omega \left\vert \dfrac{\mathrm{d}\sigma}{\mathrm{d} \Omega} \right\vert \left\vert v_1-v_2 \right\vert \left[ f_1(p_1,q_1;t)\cdot f_1(p_2,q_1;t)-f_1(p_1',q_1';t)\cdot f_1(p_2',q_1;t) \right] 
    \end{equation}
        
    


    
    